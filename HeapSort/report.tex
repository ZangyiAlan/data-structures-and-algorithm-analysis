\documentclass[UTF8]{ctexart}
\usepackage{geometry, CJKutf8}
\geometry{margin=1.5cm, vmargin={0pt,1cm}}
\setlength{\topmargin}{-1cm}
\setlength{\paperheight}{29.7cm}
\setlength{\textheight}{25.3cm}

\usepackage{amsfonts}
\usepackage{amsmath}
\usepackage{amssymb}
\usepackage{amsthm}
\usepackage{enumerate}
\usepackage{graphicx}
\usepackage{multicol}
\usepackage{fancyhdr}
\usepackage{layout}
\usepackage{listings}
\usepackage{float, caption}

\lstset{
    basicstyle=\ttfamily, basewidth=0.5em
}

% some common command
\newcommand{\dif}{\mathrm{d}}
\newcommand{\avg}[1]{\left\langle #1 \right\rangle}
\newcommand{\difFrac}[2]{\frac{\dif #1}{\dif #2}}
\newcommand{\pdfFrac}[2]{\frac{\partial #1}{\partial #2}}
\newcommand{\OFL}{\mathrm{OFL}}
\newcommand{\UFL}{\mathrm{UFL}}
\newcommand{\fl}{\mathrm{fl}}
\newcommand{\op}{\odot}
\newcommand{\Eabs}{E_{\mathrm{abs}}}
\newcommand{\Erel}{E_{\mathrm{rel}}}

\begin{document}

\pagestyle{fancy}
\fancyhead{}
\chead{堆排序算法实现与性能分析}
\rhead{2024年12月1日}

\section{设计思路}

本次实现了基于最大堆的堆排序算法。主要包含两个核心函数:

\begin{enumerate}
    \item heapify:用于维护最大堆性质的核心函数
    \item heapsort:主排序函数
\end{enumerate}

\subsection{核心函数实现}

\begin{itemize}
    \item heapify函数实现了"下沉"操作,确保以某个节点为根的子树满足最大堆性质
    \item heapsort函数分两个阶段:
        \begin{enumerate}
            \item 建堆阶段:自底向上构建最大堆
            \item 排序阶段:反复取出堆顶元素并重建堆
        \end{enumerate}
\end{itemize}

\section{测试方案}

测试程序设计了四种不同的输入序列:
\begin{itemize}
    \item 随机序列:使用随机数生成器生成
    \item 有序序列:已经排好序的序列
    \item 逆序序列:完全逆序的序列
    \item 部分重复序列:有限范围内的随机数,确保存在重复元素
\end{itemize}

每种序列的测试规模均为100万个元素。

\section{性能对比}

\begin{table}[h]
\centering
\begin{tabular}{|l|c|c|}
\hline
序列类型 & my heapsort time(ms) & std::sort\_heap(ms) \\
\hline
随机序列 & 83 & 81 \\
有序序列 & 51 & 33 \\
逆序序列 & 50 & 37 \\
部分重复序列 & 90 & 80 \\
\hline
\end{tabular}
\caption{不同序列类型的排序性能对比}
\end{table}

\section{复杂度分析}

\subsection{时间复杂度}
\begin{itemize}
    \item heapify操作:O(log n)
    \item 建堆阶段:O(n)
    \item 排序阶段:O(n log n)
\end{itemize}

整体时间复杂度为O(n log n)。

\subsection{与std::sort\_heap的效率差异分析}

从测试结果可以看出:
\begin{enumerate}
    \item 在随机序列和部分重复序列上,我的实现与标准库性能接近
    \item 在有序和逆序序列上,标准库实现显著优于我的实现
    \item 性能差异的主要原因:
        \begin{itemize}
            \item 我的实现使用递归的heapify,而标准库使用迭代实现
        \end{itemize}
\end{enumerate}

\end{document}