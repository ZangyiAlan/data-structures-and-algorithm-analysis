\documentclass[UTF8]{ctexart}
\usepackage{geometry, CJKutf8}
\geometry{margin=1.5cm, vmargin={0pt,1cm}}
\setlength{\topmargin}{-1cm}
\setlength{\paperheight}{29.7cm}
\setlength{\textheight}{25.3cm}

% useful packages.
\usepackage{amsfonts}
\usepackage{amsmath}
\usepackage{amssymb}
\usepackage{amsthm}
\usepackage{enumerate}
\usepackage{graphicx}
\usepackage{multicol}
\usepackage{fancyhdr}
\usepackage{layout}
\usepackage{listings}
\usepackage{float, caption}

\lstset{
    basicstyle=\ttfamily, basewidth=0.5em
}

\begin{document}

\pagestyle{fancy}
\fancyhead{}
\chead{数据结构与算法第六次作业}
\rhead{Nov.11th, 2024}

\section{AVL树的删除操作实现}

在实现AVL树的删除操作时,我采用了以下策略:

\subsection{基本删除流程}

删除操作分为三种情况处理:
\begin{enumerate}
    \item 被删除节点是叶子节点或只有一个子节点:直接删除并重新连接
    \item 被删除节点有两个子节点:
        \begin{itemize}
            \item 使用detachMin函数找到右子树中的最小节点
            \item 将该最小节点替换到被删除节点的位置
            \item 保持原有的左右子树连接关系
        \end{itemize}
\end{enumerate}

\subsection{平衡维护}

删除节点后,需要自底向上维护AVL树的平衡性:

\begin{enumerate}
    \item 更新节点高度:计算左右子树的最大高度+1
    \item 计算平衡因子:左子树高度减右子树高度
    \item 根据平衡因子进行旋转操作:
        \begin{itemize}
            \item 左左情况(balance > 1 且左子树左偏):右旋
            \item 右右情况(balance < -1 且右子树右偏):左旋
            \item 左右情况(balance > 1 且左子树右偏):先左旋后右旋
            \item 右左情况(balance < -1 且右子树左偏):先右旋后左旋
        \end{itemize}
\end{enumerate}

\subsection{实现优化}

为了提高效率,我做了以下优化:
\begin{enumerate}
    \item 使用detachMin而不是findMin,避免多余的节点复制
    \item 在旋转操作中直接更新高度,减少重复计算
    \item 仅在必要时进行平衡操作,避免不必要的旋转
\end{enumerate}

通过这种实现方式,确保了删除操作后树仍然保持AVL平衡特性,即任意节点的左右子树高度差不超过1。同时,通过优化的实现方式,保证了操作的效率。

\end{document}