\documentclass[UTF8]{ctexart}
\usepackage{geometry, CJKutf8}
\geometry{margin=1.5cm, vmargin={0pt,1cm}}
\setlength{\topmargin}{-1cm}
\setlength{\paperheight}{29.7cm}
\setlength{\textheight}{25.3cm}

% useful packages.
\usepackage{amsfonts}
\usepackage{amsmath}
\usepackage{amssymb}
\usepackage{amsthm}
\usepackage{enumerate}
\usepackage{graphicx}
\usepackage{multicol}
\usepackage{fancyhdr}
\usepackage{layout}
\usepackage{listings}
\usepackage{float, caption}

\lstset{
    basicstyle=\ttfamily, basewidth=0.5em,
    breaklines=true,
    numbers=left,
    numberstyle=\tiny,
    frame=single,
    showstringspaces=false
}

\begin{document}

\pagestyle{fancy}
\fancyhead{}
\lhead{姓名:臧熠, 学号:3230102001}
\chead{二叉搜索树删除操作测试报告}
\rhead{\today}

\section{测试程序的设计思路}

本测试程序主要针对二叉搜索树的删除操作进行全面测试,测试用例的设计覆盖了以下几个主要方面:

\begin{enumerate}
    \item 基础功能测试
    \begin{itemize}
        \item 空树的删除操作
        \item 叶子节点的删除
        \item 只有一个子节点的节点删除
        \item 有两个子节点的节点删除
    \end{itemize}
    
    \item 特殊情况测试
    \begin{itemize}
        \item 删除不存在的节点
        \item 连续右子树的情况
        \item 连续左子树的情况
    \end{itemize}
    
    \item 完整性测试
    \begin{itemize}
        \item 删除所有节点
        \item 树的空状态验证
    \end{itemize}
\end{enumerate}

\section{测试程序实现}

测试程序的核心代码如下:

\begin{lstlisting}[language=C++]
int main() {
    BinarySearchTree<int> bst;

    // 测试1:删除空树中的节点
    bst.remove(10);

    // 测试2:构建基本测试树
    bst.insert(50);  // 根节点
    bst.insert(30);  // 左子树
    bst.insert(70);  // 右子树
    bst.insert(20);  // 左左子树
    bst.insert(40);  // 左右子树
    bst.insert(60);  // 右左子树
    bst.insert(80);  // 右右子树

    // 测试3:删除叶子节点
    bst.remove(20);

    // 测试4:删除只有一个子节点的节点
    bst.remove(30);

    // 测试5:删除有两个子节点的节点
    bst.remove(50);

    // 后续测试...
}
\end{lstlisting}

\section{测试结果分析}

\subsection{基本功能测试结果}

\begin{enumerate}
    \item 空树测试(测试1)
    \begin{lstlisting}
=== 测试1:删除空树中的节点 ===
树的内容:
Empty tree
是否为空:是
    \end{lstlisting}
    结果显示程序正确处理了空树的情况。

    \item 树的构建(测试2)
    \begin{lstlisting}
=== 测试2:插入节点构建树 ===
树的内容:
20 30 40 50 60 70 80
是否为空:否
    \end{lstlisting}
    中序遍历结果显示树的构建符合二叉搜索树的性质。

    \item 删除操作测试(测试3-5)
    \begin{itemize}
        \item 叶子节点的删除(测试3)成功移除了节点20
        \item 单子节点的删除(测试4)成功处理了节点30的删除
        \item 双子节点的删除(测试5)正确处理了节点50的删除并保持了树的结构
    \end{itemize}
\end{enumerate}

\subsection{特殊情况测试结果}

\begin{enumerate}
    \item 删除不存在节点(测试6)
    \begin{lstlisting}
=== 测试6:删除不存在的节点(100) ===
树的内容:
40 60 70 80
是否为空:否
    \end{lstlisting}
    结果显示树的结构未受影响。

    \item 特殊树结构测试(测试8-9)
    \begin{lstlisting}
=== 测试8:特殊情况 - 连续右子树 ===
树的内容:
10 20 30
...
=== 测试9:特殊情况 - 连续左子树 ===
树的内容:
10 20 30
    \end{lstlisting}
    结果显示程序能正确处理极端的树结构情况。
\end{enumerate}

\section{改进建议}

基于测试结果,提出以下改进建议:

\begin{enumerate}
    \item 添加层次遍历输出功能,以更直观地显示树的结构
    \item 在删除操作后增加查找测试,验证树的查找功能完整性
    \item 添加树高度验证,确保删除操作后树的平衡性
    \item 增加更多边界测试用例
\end{enumerate}

\section{结论}

通过全面的测试,我们可以得出以下结论:

\begin{enumerate}
    \item 删除操作能正确处理所有基本情况
    \item 边界条件(空树、不存在节点)被正确处理
    \item 特殊的树结构(连续左/右子树)也能正确处理
    \item 删除操作后树始终保持二叉搜索树的性质
\end{enumerate}

测试结果表明二叉搜索树的删除操作实现是正确和可靠的。

\end{document}